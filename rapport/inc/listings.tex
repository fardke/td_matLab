% listings package config file
% generated on Tue Jan 04 15:00:38 +0100 2011
% by Kévin Fardel et Rick Ghanem
% With the Condom lib (version 1.1.0)
% See http://github.com/v0n/condom

% /!\ ne pas utiliser de caracteres accentues dans les sources (ne gere pas l'utf-8)
% pour remplacer les lettres accentues : sed -i -e "y/ÉÈÊÇÀÔéèçàôîêûùï/EEECAOeecaoieuui/" fichier

% configuration des listings par defaut :
\lstset{ %
	language={},                % choose the language of the code
	basicstyle=\footnotesize,       % the size of the fonts that are used for the code
	numbers=left,                   % where to put the line-numbers
	numberstyle=\footnotesize,      % the size of the fonts that are used for the line-numbers
	stepnumber=2,                   % the step between two line-numbers. If it's 1 each line 
	numbersep=5pt,                  % how far the line-numbers are from the code
	backgroundcolor=\color{white},  % choose the background color. You must add \usepackage{color}
	showspaces=false,               % show spaces adding particular underscores
	showstringspaces=false,         % underline spaces within strings
	showtabs=false,                 % show tabs within strings adding particular underscores
	frame=single,	                % adds a frame around the code
	tabsize=2,	                % sets default tabsize to 2 spaces
	captionpos=b,                   % sets the caption-position to bottom
	breaklines=true,                % sets automatic line breaking
	breakatwhitespace=false,        % sets if automatic breaks should only happen at whitespace
	title=\lstname,                 % show the filename of files included with \lstinputlisting;
	commentstyle=\color{blue},                                % also try caption instead of title
	keywordstyle=\color{green},
    	ndkeywordstyle=\color{darkgreen},
    	identifierstyle=\color{cyan},
    	stringstyle=\color{red},
	escapeinside={\%*}{*)},         % if you want to add a comment within your code
	morekeywords={*,...}            % if you want to add more keywords to the set
}

\lstloadlanguages{
    %[Visual]Basic
    %Pascal
    %C
    %C++
    %XML
    %HTML
    %Java
}
%\DeclareCaptionFont{blue}{\color{blue}}

%\captionsetup[lstlisting]{singlelinecheck=false, labelfont={blue}, textfont={blue}}
\DeclareCaptionFont{white}{\color{white}}
\DeclareCaptionFormat{listing}{\colorbox[cmyk]{0.43, 0.35, 0.35,0.01}{\parbox{\textwidth}{\hspace{15pt}#1#2#3}}}
\captionsetup[lstlisting]{format=listing,labelfont=white,textfont=white, singlelinecheck=false, margin=0pt, font={bf,footnotesize}}

% configuration des listings console :
\lstnewenvironment{console}
{\lstset{%
    language={},
    numbers=none,
    extendedchars=true,
    framexleftmargin=5mm,
    %float,
    showstringspaces=false,
    showspaces=false,
    showtabs=false,
    breaklines=false,
    backgroundcolor=\color{darkgray},
    basicstyle=\color{white} \scriptsize \ttfamily,
    keywordstyle=\color{white},
    ndkeywordstyle=\color{white},
    commentstyle=\color{white},
    identifierstyle=\color{white},
    stringstyle=\color{white}
}}
{}

\renewcommand{\lstlistlistingname}{Table des codes sources} % renommer la liste des listings

% un listing depuis un fichier s'importe comme ceci :
%\lstinputlisting[caption={Legende}, label=lst:label]{emplacement}

