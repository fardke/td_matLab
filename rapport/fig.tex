% Tips to insert images in LaTeX
% generated on Tue Jan 04 15:00:38 +0100 2011
% by Kévin Fardel et Rick Ghanem
% With the Condom lib (version 1.1.0)
% See http://github.com/v0n/condom

% This file is useless, it's just a note about images manipulation with LaTeX.

% [!h] est facultatif, ca permet d'indiquer la position de la figure (si possible !)
% ! = forcer la position
% h = here
% t = top
% b = bottom
% p = page separee
% si il y a un probleme avec le positionnement, il peut etre force avec la position [H] (necessite le paquet float)

\begin{figure}[!h]
\begin{center}
%\includegraphics[width=5cm]{images/freebsd.png}
%\includegraphics[height=4cm]{images/freebsd.png}
\includegraphics{images/freebsd.png}
\caption{\label{freebsd}Logo de FreeBSD}
\end{center}
\end{figure}

% images cotes a cotes avec une seule legende
% note de bas de page dans un caption (\footnote ne peut pas etre utilise)
\begin{figure}[!h]
   \begin{center}
      \includegraphics[height=2cm]{images/freebsd.png}
      \hspace{1cm} % espace horizontal (facultatif)
      \includegraphics[height=2cm]{images/openbsd.png}
      \hspace{1cm}
      \includegraphics[height=2cm]{images/netbsd.png}
   \end{center}
   \caption{FreeBSD, OpenBSD et NetBSD\protect\footnotemark}
\end{figure}
\footnotetext{Tous des systemes d'exploitation libres.}

% images cotes a cotes avec une legende pour chaque
% si 2 images, mettre comme largeur 0.5\textwidth
\begin{figure}[!h]
\begin{minipage}{0.33\textwidth}
   \begin{center}
      \includegraphics[height=2cm]{images/freebsd.png}
      \caption{FreeBSD}
   \end{center}
\end{minipage}
\begin{minipage}{0.33\textwidth}
   \begin{center}
      \includegraphics[height=2cm]{images/openbsd.png}
      \caption{OpenBSD}
   \end{center}
\end{minipage}
\begin{minipage}{0.33\textwidth}
   \begin{center}
      \includegraphics[height=2cm]{images/netbsd.png}
      \caption{NetBSD}
   \end{center}
\end{minipage}
\end{figure}

